\documentclass[a4paper]{article}
\usepackage{graphicx}
\begin{document}
\title{Latex test}
\author{Dang Quang Minh}
\date{May 28, 2018}

\maketitle

\section{Instruction for the test}
\subsection{How to do the test}
	In this test, you are asked to create a document that looks EXACTLY this document by using Latex.
	\begin{itemize}
		\item You should use ``article" for document class.
		\item Put your full name dand date in the title page by using \textsf{\textbackslash author} and \textsf{\textbackslash date} command.
		\item Use sectionning commands for creating sections in this document.
		\item Use reference commands for creating references.
		\item Use commands for creating table of content and list of figures etc. In this test, the table of contents and list of figures should be at end of the document.
	\end{itemize}

\subsection{Submitting your test}
The test will be 60 minutes length. At the end of test time you have to submit your test for professor. Create a directory which named with your full name and put your files there. For example, of your name is Nguyen Van A, so the file should be in directory Nguyen-Van-A.\\
You have to submit the following files:
\begin{enumerate}
	\item The source code of your Latex file: .tex file
	\item All the picture files that the document requires.
	\item The pdf file of the final document.
\end{enumerate}

Submit your file by using submit program with following syntax:\\
\$ ./submit path-to-your-directory

For example:\\
\$ ./submit Nguyen-Van-A

\section{Mathematical formula}
\subsection{Solving the set of equation}
	Given an equation:
	\begin{eqnarray}
		a_1 x + b_1 y & = & c_1\\
		a_2 x + b_2 y & = & c_2
	\end{eqnarray}
	Let
	\begin{equation}
		\Delta = \left[
		\begin{array}{cc}
			a_1 & b_1\\
			a_2 & b_2		
		\end{array}
		\right]
	\end{equation}
Then $det (\Delta) = a_1 b_2 - a_2 b_1$, we obtain the following results
	\begin{eqnarray}
		x & = & \frac{c_1 b_2 - c_2 b_1}{det(\Delta)}\\
		y & = & \frac{a_1 c_2 - a_2 c_1}{det(\Delta)}
	\end{eqnarray}
	
\section{Tables and tabulars}
	\begin{table}[tbh]
		\centerline{
			\begin{tabular}{|l|r|r|}
			\hline
			\textbf{Date} & \textbf{Maximum pression} & \textbf{Minimum pression}\\
			\hline\hline
			10/10/2009 & 120 & 70\\
			\hline
			11/10/2009 & 135 & 78\\
			\hline
			&&\\
			\hline
			\end{tabular}
		}
		\caption{Blood pression in 2009}
		\label{tab:blood}
	\end{table}

\section{Figures}
	Use the figure in the file shapes.pdf. Insert this figure in this page so that the figure take 60\%\ page width. Use the option width=0.6\textbackslash textwidth
	\begin{figure}[bth]
		\centerline{\includegraphics[width=0.6\textwidth]{shapes.pdf}}
		\caption{Graphics incorporation}
	\end{figure}
\section{Cross reference}
	See Table \ref{tab:blood} for an example of a table
	
\tableofcontents

\end{document}